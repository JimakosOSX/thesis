%
\documentclass[a4paper,12pt]{article}
\usepackage[utf8]{inputenc}
\usepackage[greek,english]{babel}
\usepackage{alphabeta}


\begin{document}


\title{
    %Υποδομή σαν Κώδικας και Αυτοματοποιημένη Διαχείρηση  \\
        \large Νέες πρακτικές με Ansible}

\author{Δημήτριος Μακρής}
\date{\today}
\maketitle

\renewcommand*\contentsname{Περιεχόμενα}
\tableofcontents

\section{Εισαγωγή}
Μέσα στα χρόνια, ο κλάδος του ΙΤ έχει αλλάξει - τόσο ως προς την χρήση του όσο και τη λειτουργία του.\\
Με τη βοήθεια τεχνολογιών αυτοματισμού, όπως η \textbf{Ansible} δίνεται η ευκαιρία στους ΙΤ η έννοια της υποδομής (\textbf{Infrastructure}) να απαιτεί πολύ λιγότερο χρόνο συντήρησης, αφού δεν διαχειριζόμαστε κάθε σύστημα ξεχωριστά αλλά γράφουμε εφαρμογές που φέρνουν το κάθε σύστημα στην κατάσταση που εμείς επιθυμούμε.\hfill \break
Εκεί που η Ansible ξεχωρίζει είναι στην ανάπτυξη λογισμικού αφού πλατφόρμες όπως το Gitlab, προσφέρουν τη δυνατότητα ένταξης Ansible playbooks, με τη μορφή αρχείων Συνεχόυς Ενσωμάτωσης (\textbf{Continous Integration}).\hfill \break
Ετσι, οι προγραμματιστές δημιουργούν αγωγούς ή αλλιως \textbf{pipelines} τα οποία αυτόματα μπορούν να τρέξουν μια η περισσότερες εργασίες, και να συνεχίσουν να λειτουργούν σε προδιαγραμμένο χρόνο, χωρίς την ανάγκη ανθρώπινης παρέμβασης.\hfill \break
Τ

\newpage

\section{Συνοπτική οργάνωση εργασίας}
%\hfill\break
\begin{itemize}
    \item Εισαγωγή
    \item Σχετικά εργαλεία που χρησιμοποιούνται για διαχείριση διαμόρφωσης 
    \item Ansible \\
        - Εισαγωγή\\
        - Invertory\\
        - Ad-Hoc Commands\\
        - Playbooks\\
        - Variables\\
        - Facts\\
        - Task Control\\
        - Templates\\
        - Roles\\
     Συμπεράσματα
    \item Διαχείριση Διαμόφωσης στην πράξη\\
        3 Playbooks - και πως λειτουργούν 
    \item Δοκιμές και αποτελέσματα
    \item Oι παλαιότερες πρακτικές διαχείρισης χρειάζονται ακόμα?
    \item Συμπεράσματα 
    \item Αναφορές
    \item Βιβλιογραφία
\end{itemize}
%\newpage
%\tableofcontents

\end{document}
